\section{Toma de Requerimientos}

\subsection{Preguntas y Respuestas}

1. \textbf{¿Cómo se lleva a cabo la evaluación de los trabajos terminales actualmente? ¿Qué limitantes tiene?}
   \begin{itemize}
       \item \textit{Respuesta:} Se lleva el registro y control mediante correos y hojas de Excel.
   \end{itemize}

2. \textbf{¿Qué roles y responsabilidades tendrán los administradores en este sistema? ¿Qué acciones específicas podrán realizar?}
   \begin{itemize}
       \item \textit{Respuesta:} Los administradores podrán dar de alta usuarios y registrar hasta dos trabajos terminales por usuario.
   \end{itemize}

3. \textbf{¿Cuáles son las funciones y capacidades del alumno y profesor?}
   \begin{itemize}
       \item \textit{Respuesta Alumno:} Podrá ver la fecha de presentación, el profesor asignado y sus horarios de consulta.
       \item \textit{Respuesta Profesor:} Podrá consultar y subir la calificación final en un máximo de tres días.
   \end{itemize}

4. \textbf{¿Cómo será la retroalimentación de los profesores?}
   \begin{itemize}
       \item \textit{Respuesta:} La retroalimentación se realizará mediante comentarios.
   \end{itemize}

5. \textbf{¿Qué métricas usarán los profesores para la evaluación?}
   \begin{itemize}
       \item \textit{Respuesta:} No se usarán métricas, se calificarán avances.
   \end{itemize}

6. \textbf{¿Cómo es la interacción entre equipos de alumnos?}
   \begin{itemize}
       \item \textit{Respuesta:} Se permiten equipos con un máximo de cuatro alumnos.
   \end{itemize}

7. \textbf{¿Qué datos de los profesores y alumnos necesito?}
   \begin{itemize}
       \item \textit{Respuesta Alumnos:} Nombre, boleta, materia necesaria para acreditar el trabajo terminal.
       \item \textit{Respuesta Profesores:} Nombre, departamento, número de empleado (no visible).
   \end{itemize}

8. \textbf{¿Interesa guardar los trabajos terminales?}
   \begin{itemize}
       \item \textit{Respuesta:} No, únicamente es necesario saber si ya se entregó.
   \end{itemize}

9. \textbf{¿Qué tipo de plataforma se requiere para el sistema?}
   \begin{itemize}
       \item \textit{Respuesta:} Se necesita una plataforma web compatible con PHP.
   \end{itemize}

10. \textbf{¿Cómo se realizará el inicio de sesión?}
   \begin{itemize}
       \item \textit{Respuesta Alumnos:} A través de boleta, CURP y correo electrónico.
       \item \textit{Respuesta Profesores:} Con el número de empleado y la CURP.
   \end{itemize}

11. \textbf{¿Cómo se integrarán los datos con la plataforma SAES?}
   \begin{itemize}
       \item \textit{Respuesta:} Los datos se obtendrán mediante la plataforma SAES utilizando la boleta y el número de empleado.
   \end{itemize}

12. \textbf{¿Qué tipo de base de datos se utilizará en el sistema?}
   \begin{itemize}
       \item \textit{Respuesta:} Se utilizará una base de datos relacional.
   \end{itemize}

13. \textbf{¿Cuál es el presupuesto asignado para el proyecto?}
   \begin{itemize}
       \item \textit{Respuesta:} El presupuesto asignado es de 2.5 millones de pesos.
   \end{itemize}

\subsection{Requisitos del Sistema}

% Aquí irían las tablas de requisitos funcionales y no funcionales.
